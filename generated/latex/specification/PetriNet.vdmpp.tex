\begin{vdmpp}
class PetriNet
types
 public Marking = map Place to nat;
instance variables
 public Places : set of Place := {};
 public Transitions : set of Transition := {};
 public InitialMarking : Marking  := {|->};
 public CurrentMarking : Marking := {|->};
 
 -- Checks if all inputs, outputs, resets and inhibitors in every transition belong to the set of places in this Petri net.
 inv forall t in set Transitions & dom t.Inputs subset Places and
                  dom t.Outputs subset Places and
                  t.Resets subset Places and
                  t.Inhibitors subset Places;

 
operations


 -- Constructor for the Petri Net.
 -- @param places: the set of places in the net
 -- @param transitions: the set of transitions in the net
 -- @returns the petri net
(*@
\label{PetriNet:24}
@*)
 public PetriNet(places: set of Place, transitions: set of Transition) result : PetriNet == (
      Places := places;
      Transitions := transitions;
      InitialMarking := {|->};
      CurrentMarking := InitialMarking;
      
      return self;    
 )
 pre forall t in set Transitions & dom t.Inputs subset Places and
                  dom t.Outputs subset Places
 post result.Places = places and result.Transitions = transitions and 
        result.InitialMarking = {|->} and result.CurrentMarking = {|->};
     
        
        
 -- Checks if a transition is enabled at the time.
 -- @param t: the transition to check
 -- @returns true if the transition is enabled, false otherwise
(*@
\label{transitionEnabled:42}
@*)
 public transitionEnabled(t : Transition) result : bool == (
  dcl enabled: bool := true;
  
  for all i in set dom t.Inputs do (
   if CurrentMarking(i) < t.Inputs(i) then enabled := false;
  );
  
  for all i in set t.Inhibitors do (
   if CurrentMarking(i) > 0 then enabled := false;
  );
  
  return enabled;
 )
 pre t in set Transitions;
 
 
 
 -- Gets all enabled transitions in the net.
 -- @returns a set containing all enabled transitions at the time
(*@
\label{getEnabled:61}
@*)
 public getEnabled() result : set of Transition == (
  dcl enabledTrans: set of Transition := {};
  
  for all t in set Transitions do 
   if transitionEnabled(t) then enabledTrans := enabledTrans union {t};
   
  return enabledTrans;
 )
 post result subset Transitions and
    (forall t in set result & forall i in set dom t.Inputs & CurrentMarking(i) = t.Inputs(i)) and
    (forall t in set result & forall i in set t.Inhibitors & CurrentMarking(i) = 0);
 
 
 
 -- Resets the petri net marking to its initial marking.
(*@
\label{reset:76}
@*)
 public reset() == (
  CurrentMarking := InitialMarking;
 )
 post CurrentMarking = InitialMarking;
 
 
 
 -- Triggers the selected transition if it's enabled.
 -- @param t: The transition to trigger.
(*@
\label{fireTransition:85}
@*)
 public fireTransition(t : Transition) == (
  if(transitionEnabled(t)) then (
   for all i in set dom t.Inputs do
    CurrentMarking := CurrentMarking ++ {i |-> CurrentMarking(i) - t.Inputs(i)};
  
   for all o in set dom t.Outputs do
    CurrentMarking := CurrentMarking ++ {o |-> CurrentMarking(o) + t.Outputs(o)};
    
   for all r in set t.Resets do
    CurrentMarking := CurrentMarking ++ {r |-> 0};
  );
 )
 pre t in set Transitions and 
    (forall i in set dom t.Inputs & CurrentMarking(i) = t.Inputs(i)) and
    (forall i in set t.Inhibitors & CurrentMarking(i) = 0)
 post (forall p in set dom t.Inputs & CurrentMarking(p) = CurrentMarking~(p) - t.Inputs(p)) and
    (forall p in set dom t.Outputs & CurrentMarking(p) = CurrentMarking~(p) + t.Outputs(p));
    
 
 -- Calculates the reachability graph of this Petri net.
 -- @returns the reachability graph of this petry net.
(*@
\label{calculateReachabilityGraph:106}
@*)
 public calculateReachabilityGraph() result : ReachabilityGraph == (
  dcl Graph : ReachabilityGraph := new ReachabilityGraph();
  dcl Work : seq of Marking := [InitialMarking];
  dcl M : Marking;
  dcl newM : Marking;
  dcl edge : Edge;
  dcl pn : PetriNet := new PetriNet(Places,Transitions);
  pn.InitialMarking := InitialMarking;
  
  Graph.Vertexes := {InitialMarking};
  Graph.Edges := {};
  Graph.InitialV := InitialMarking;
  
  while Work <> [] do (
   M := hd Work;
   Work := tl Work;
   
   pn.CurrentMarking := M;
   for all t in set pn.getEnabled() do (
    pn.fireTransition(t);
    newM := pn.CurrentMarking;
    if newM not in set Graph.Vertexes then (
     Graph.Vertexes := Graph.Vertexes union {newM};
     Work := Work ^ [newM];
    );
    
    edge := new Edge();
    edge.Origin := M;
    edge.Result := newM;
    edge.T := t;
    
    Graph.Edges := Graph.Edges union {edge};
   );
  );
  
  return Graph;
 )
 post (forall m in set result.Vertexes & dom m subset Places) and 
    result.InitialV = InitialMarking and
    (forall e in set result.Edges & dom e.Origin subset Places and dom e.Result subset Places and e.T in set Transitions);
 
 
 -- Calculates if a given marking is reachable from the initial marking on this Petri net.
 -- @param m: the marking to calculate the reachability for.
 -- @returns true if m is reachable, false otherwise.
(*@
\label{isMarkingReachable:151}
@*)
 public isMarkingReachable(m : Marking) result : bool == (
  dcl Graph : ReachabilityGraph := calculateReachabilityGraph();
  if m in set Graph.Vertexes then return true;
  
  return false;
 )
 pre dom m subset Places;
 
 
 
 -- Calculates the set of reachable markings from the initial marking in this Petri net.
 -- @returns the set of reachable markings.
(*@
\label{getReachableMarkings:163}
@*)
 public getReachableMarkings() result : set of Marking == (
  dcl Graph : ReachabilityGraph := calculateReachabilityGraph();
  return Graph.Vertexes;
 )
 post forall m in set result & dom m subset Places;

end PetriNet
\end{vdmpp}
\bigskip
\begin{longtable}{|l|r|r|r|}
\hline
Function or operation & Line & Coverage & Calls \\
\hline
\hline
\hyperref[PetriNet:24]{PetriNet} & 24&0.0\% & 0 \\
\hline
\hyperref[calculateReachabilityGraph:106]{calculateReachabilityGraph} & 106&0.0\% & 0 \\
\hline
\hyperref[fireTransition:85]{fireTransition} & 85&0.0\% & 0 \\
\hline
\hyperref[getEnabled:61]{getEnabled} & 61&0.0\% & 0 \\
\hline
\hyperref[getReachableMarkings:163]{getReachableMarkings} & 163&0.0\% & 0 \\
\hline
\hyperref[isMarkingReachable:151]{isMarkingReachable} & 151&0.0\% & 0 \\
\hline
\hyperref[reset:76]{reset} & 76&0.0\% & 0 \\
\hline
\hyperref[transitionEnabled:42]{transitionEnabled} & 42&0.0\% & 0 \\
\hline
\hline
PetriNet.vdmpp & & 0.0\% & 0 \\
\hline
\end{longtable}

