\begin{vdmpp}
class TestPetriNet
types
-- TODO Define types here
values
-- TODO Define values here
instance variables
-- TODO Define instance variables here
operations
(*@
\label{assertTrue:9}
@*)
 private assertTrue: bool ==> ()
  assertTrue(cond) == return
 pre cond;

(*@
\label{RunTest1:13}
@*)
 public RunTest1() == (
  dcl p1 : Place := new Place();
  dcl p2 : Place := new Place();
  dcl p3 : Place := new Place();
  dcl p4 : Place := new Place();
  
  dcl t1 : Transition := new Transition({p1 |-> 1, p2 |-> 1},{p3 |-> 1},{},{});
  dcl t2 : Transition := new Transition({p3 |-> 1}, {p4 |-> 2},{},{});
  
  dcl pn1 : PetriNet := new PetriNet({p1,p2,p3,p4},{t1,t2});
  
  p3.Id := 3;
  p1.Id := 1;
  p2.Id := 2;
  p4.Id := 4;
  
  pn1.InitialMarking := {p1 |-> 1, p2 |-> 1, p3 |-> 0, p4 |-> 0};
  pn1.CurrentMarking := {p1 |-> 1, p2 |-> 1, p3 |-> 0, p4 |-> 0};
  
  assertTrue(pn1.transitionEnabled(t1) = true);
  assertTrue(pn1.transitionEnabled(t2) = false);
  
  pn1.fireTransition(t1);
  
  assertTrue(pn1.transitionEnabled(t2) = true);
  
  pn1.fireTransition(t2);
  
  assertTrue(pn1.CurrentMarking(p4) = 2);
 );
 
(*@
\label{RunTest2:44}
@*)
 public RunTest2() == (
  dcl p1 : Place := new Place();
  dcl p2 : Place := new Place();
  dcl p3 : Place := new Place();
  dcl p4 : Place := new Place();
  
  dcl t1 : Transition := new Transition({|->},{p3 |-> 1},{p1},{p2});
  dcl t2 : Transition := new Transition({|->}, {p4 |-> 2},{},{p3}); -- Can a transition fire if its only input is an inhibitor? Must check this...
  
  dcl pn1 : PetriNet := new PetriNet({p1,p2,p3,p4},{t1,t2});
  
  p3.Id := 3;
  p1.Id := 1;
  p2.Id := 2;
  p4.Id := 4;
  
  pn1.InitialMarking := {p1 |-> 1, p2 |-> 0, p3 |-> 0, p4 |-> 0};
  pn1.CurrentMarking := {p1 |-> 1, p2 |-> 0, p3 |-> 0, p4 |-> 0};
  
  assertTrue(pn1.getEnabled() = {t1,t2});
  
  pn1.fireTransition(t1);
  
  assertTrue(pn1.CurrentMarking(p1) = 0);
  assertTrue(pn1.transitionEnabled(t2) = false);
  
  pn1.reset();
  
  assertTrue(pn1.CurrentMarking = {p1 |-> 1, p2 |-> 0, p3 |-> 0, p4 |-> 0});
  
 );
 
(*@
\label{RunTest3:76}
@*)
 public RunTest3() == (
  dcl p1 : Place := new Place();
  dcl p2 : Place := new Place();
  dcl p3 : Place := new Place();
  dcl p4 : Place := new Place();
  
  dcl t1 : Transition := new Transition({p1 |-> 1, p2 |-> 1},{p3 |-> 1},{},{});
  dcl t2 : Transition := new Transition({p3 |-> 1}, {p4 |-> 2},{},{});
  
  dcl pn1 : PetriNet := new PetriNet({p1,p2,p3,p4},{t1,t2});
  
  dcl reachable : set of PetriNet`Marking;
  
  p3.Id := 3;
  p1.Id := 1;
  p2.Id := 2;
  p4.Id := 4;
  
  pn1.InitialMarking := {p1 |-> 1, p2 |-> 1, p3 |-> 0, p4 |-> 0};
  pn1.CurrentMarking := {p1 |-> 1, p2 |-> 1, p3 |-> 0, p4 |-> 0};
  
  reachable := pn1.getReachableMarkings();
  
  assertTrue(reachable subset { {p1 |-> 1, p2 |-> 1, p3 |-> 0, p4 |-> 0} , {p1 |-> 0, p2 |-> 0, p3 |-> 1, p4 |-> 0}, {p1 |-> 0, p2 |-> 0, p3 |-> 0, p4 |-> 2}});
  assertTrue(pn1.isMarkingReachable({p1 |-> 0, p2 |-> 0, p3 |-> 0, p4 |-> 2}));
  assertTrue(pn1.isMarkingReachable({p1 |-> 0, p2 |-> 0, p3 |-> 0, p4 |-> 4}) = false);
 );
 
(*@
\label{main:104}
@*)
 public static main: () ==> ()
 main() == (
  new TestPetriNet().RunTest1();
  new TestPetriNet().RunTest2();
  new TestPetriNet().RunTest3();
 );
end TestPetriNet
\end{vdmpp}
\bigskip
\begin{longtable}{|l|r|r|r|}
\hline
Function or operation & Line & Coverage & Calls \\
\hline
\hline
\hyperref[RunTest1:13]{RunTest1} & 13&0.0\% & 0 \\
\hline
\hyperref[RunTest2:44]{RunTest2} & 44&0.0\% & 0 \\
\hline
\hyperref[RunTest3:76]{RunTest3} & 76&0.0\% & 0 \\
\hline
\hyperref[assertTrue:9]{assertTrue} & 9&0.0\% & 0 \\
\hline
\hyperref[main:104]{main} & 104&0.0\% & 0 \\
\hline
\hline
TestPetriNet.vdmpp & & 0.0\% & 0 \\
\hline
\end{longtable}

